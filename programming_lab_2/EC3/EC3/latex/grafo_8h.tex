\section{Escritorio/EC3/src/grafo.h File Reference}
\label{grafo_8h}\index{Escritorio/EC3/src/grafo.h@{Escritorio/EC3/src/grafo.h}}
Declaracion de la clase grafo. 

{\tt \#include $<$iostream$>$}\par
{\tt \#include $<$cstdlib$>$}\par
{\tt \#include $<$queue$>$}\par
\subsection*{Classes}
\begin{CompactItemize}
\item 
struct {\bf TipoArco}
\begin{CompactList}\small\item\em Tipo que representa un arco, con su origen \char`\"{}x\char`\"{} y su destino \char`\"{}y\char`\"{}. \item\end{CompactList}\item 
class {\bf Grafo}
\begin{CompactList}\small\item\em Esta clase define un grafo no dirigido valuado. \item\end{CompactList}\end{CompactItemize}
\subsection*{Defines}
\begin{CompactItemize}
\item 
\#define {\bf kDepurarGrafos}~0
\item 
\#define {\bf DEPURAR\_\-GRAFO\_\-MSG}(msg)~if (kDepurarGrafos)\{ cout$<$$<$msg; cin.get();\}
\end{CompactItemize}
\subsection*{Typedefs}
\begin{CompactItemize}
\item 
typedef int {\bf TipoNodoGrafo}
\begin{CompactList}\small\item\em Tipo referente a los nodos del grafo.(entero). \item\end{CompactList}\item 
typedef int {\bf TipoArcoGrafo}
\begin{CompactList}\small\item\em Tipo referente al valor de los arcos del grafo.(entero). \item\end{CompactList}\item 
typedef {\bf TipoArcoGrafo} {\bf TipoMatrizAdyacencia} [{\bf kMaxVert}][{\bf kMaxVert}]
\begin{CompactList}\small\item\em Matriz de tamaño kMaxVer x kMaxVert de tipo TipoArcoGrafo que se usara para representar la matriz de adyacencia del grafo. \item\end{CompactList}\item 
typedef bool {\bf TipoMatrizBooleana} [{\bf kMaxVert}][{\bf kMaxVert}]
\begin{CompactList}\small\item\em Matriz de bool de tamaño kMaxVert x kMaxVert. \item\end{CompactList}\item 
typedef {\bf TipoNodoGrafo} {\bf TipoVectorNodos} [{\bf kMaxVert}]
\begin{CompactList}\small\item\em Tipo que representa un vector estatico de nodos de tamaño kMaxVert. \item\end{CompactList}\item 
typedef queue$<$ {\bf TipoNodoGrafo} $>$ {\bf TipoCjtoNodos}
\begin{CompactList}\small\item\em Cola para almacenar nodos. \item\end{CompactList}\end{CompactItemize}
\subsection*{Functions}
\begin{CompactItemize}
\item 
int {\bf Maximo} (const int \&a, const int \&b)
\item 
int {\bf Minimo} (const int \&a, const int \&b)
\end{CompactItemize}
\subsection*{Variables}
\begin{CompactItemize}
\item 
const int {\bf kMaxVert} = 300
\begin{CompactList}\small\item\em Constante con el numero maximo de vertices que tendra el grafo. \item\end{CompactList}\item 
const int {\bf kInfinito} = 9999
\begin{CompactList}\small\item\em Constante para tratar el valor 9999 como infinito. \item\end{CompactList}\item 
const int {\bf kNoValor} = -1
\begin{CompactList}\small\item\em Constante para tratar el valor -1 como ausencia de valor. \item\end{CompactList}\end{CompactItemize}


\subsection{Detailed Description}
Declaracion de la clase grafo. 

\begin{Desc}
\item[Date:]11-04-09 \end{Desc}
\begin{Desc}
\item[Author:]{\bf Nombre:} Mario \par
 {\bf Apellidos:} Corchero Jimenez \par
 {\bf Asignatura} Laboratorio de Programacion II \par
 {\bf Curso} 08/09 \end{Desc}


\subsection{Define Documentation}
\index{grafo.h@{grafo.h}!DEPURAR_GRAFO_MSG@{DEPURAR\_\-GRAFO\_\-MSG}}
\index{DEPURAR_GRAFO_MSG@{DEPURAR\_\-GRAFO\_\-MSG}!grafo.h@{grafo.h}}
\subsubsection{\setlength{\rightskip}{0pt plus 5cm}\#define DEPURAR\_\-GRAFO\_\-MSG(msg)~if (kDepurarGrafos)\{ cout$<$$<$msg; cin.get();\}}\label{grafo_8h_9cb09aaf7c006d88cdb669d68c9ae7cf}


Macro utilizada para mostrar mensajes de depuracion de programa y retener los mensajes en pantalla \index{grafo.h@{grafo.h}!kDepurarGrafos@{kDepurarGrafos}}
\index{kDepurarGrafos@{kDepurarGrafos}!grafo.h@{grafo.h}}
\subsubsection{\setlength{\rightskip}{0pt plus 5cm}\#define kDepurarGrafos~0}\label{grafo_8h_fa8a7967855e3d2edac83a77459ac60e}


Variable constante utilizada para mostrar o no mensajes de depuracion de programa 

\subsection{Typedef Documentation}
\index{grafo.h@{grafo.h}!TipoArcoGrafo@{TipoArcoGrafo}}
\index{TipoArcoGrafo@{TipoArcoGrafo}!grafo.h@{grafo.h}}
\subsubsection{\setlength{\rightskip}{0pt plus 5cm}{\bf TipoArcoGrafo}}\label{grafo_8h_100ccbbce021b9022905e74b443960e3}


Tipo referente al valor de los arcos del grafo.(entero). 

\index{grafo.h@{grafo.h}!TipoCjtoNodos@{TipoCjtoNodos}}
\index{TipoCjtoNodos@{TipoCjtoNodos}!grafo.h@{grafo.h}}
\subsubsection{\setlength{\rightskip}{0pt plus 5cm}{\bf TipoCjtoNodos}}\label{grafo_8h_f3cc000f88d2566ca33dd725b2d7659f}


Cola para almacenar nodos. 

\index{grafo.h@{grafo.h}!TipoMatrizAdyacencia@{TipoMatrizAdyacencia}}
\index{TipoMatrizAdyacencia@{TipoMatrizAdyacencia}!grafo.h@{grafo.h}}
\subsubsection{\setlength{\rightskip}{0pt plus 5cm}{\bf TipoMatrizAdyacencia}}\label{grafo_8h_cb0e423c3f696ff8a83568aae94060d3}


Matriz de tamaño kMaxVer x kMaxVert de tipo TipoArcoGrafo que se usara para representar la matriz de adyacencia del grafo. 

\index{grafo.h@{grafo.h}!TipoMatrizBooleana@{TipoMatrizBooleana}}
\index{TipoMatrizBooleana@{TipoMatrizBooleana}!grafo.h@{grafo.h}}
\subsubsection{\setlength{\rightskip}{0pt plus 5cm}{\bf TipoMatrizBooleana}}\label{grafo_8h_61958ae5e333bed0ba9acbae8239c3e8}


Matriz de bool de tamaño kMaxVert x kMaxVert. 

\index{grafo.h@{grafo.h}!TipoNodoGrafo@{TipoNodoGrafo}}
\index{TipoNodoGrafo@{TipoNodoGrafo}!grafo.h@{grafo.h}}
\subsubsection{\setlength{\rightskip}{0pt plus 5cm}{\bf TipoNodoGrafo}}\label{grafo_8h_8e423b2f186557a37affc95091e047cb}


Tipo referente a los nodos del grafo.(entero). 

\index{grafo.h@{grafo.h}!TipoVectorNodos@{TipoVectorNodos}}
\index{TipoVectorNodos@{TipoVectorNodos}!grafo.h@{grafo.h}}
\subsubsection{\setlength{\rightskip}{0pt plus 5cm}{\bf TipoVectorNodos}}\label{grafo_8h_e05f8d1796227eaf9027828807e77936}


Tipo que representa un vector estatico de nodos de tamaño kMaxVert. 



\subsection{Function Documentation}
\index{grafo.h@{grafo.h}!Maximo@{Maximo}}
\index{Maximo@{Maximo}!grafo.h@{grafo.h}}
\subsubsection{\setlength{\rightskip}{0pt plus 5cm}int Maximo (const int \& {\em a}, const int \& {\em b})}\label{grafo_8h_bd67fd90183cf2cea5f41d88b098ba97}


Funcion auxiliar que halla el máximo de dos enteros \begin{Desc}
\item[Parameters:]
\begin{description}
\item[{\em a}]primer entero \item[{\em b}]segundo entero \end{description}
\end{Desc}
\begin{Desc}
\item[Returns:]maximo de los 2, el segundo si son iguales \end{Desc}
\index{grafo.h@{grafo.h}!Minimo@{Minimo}}
\index{Minimo@{Minimo}!grafo.h@{grafo.h}}
\subsubsection{\setlength{\rightskip}{0pt plus 5cm}int Minimo (const int \& {\em a}, const int \& {\em b})}\label{grafo_8h_ad08456ae6cb5f9aaaee75537f6ef90a}


Funcion auxiliar que halla el mínimo de dos enteros \begin{Desc}
\item[Parameters:]
\begin{description}
\item[{\em a}]primer entero \item[{\em b}]segundo entero \end{description}
\end{Desc}
\begin{Desc}
\item[Returns:]minimo de los 2, el segundo si son iguales \end{Desc}


\subsection{Variable Documentation}
\index{grafo.h@{grafo.h}!kInfinito@{kInfinito}}
\index{kInfinito@{kInfinito}!grafo.h@{grafo.h}}
\subsubsection{\setlength{\rightskip}{0pt plus 5cm}{\bf kInfinito} = 9999}\label{grafo_8h_e6b53b2e25c3356e411efd5b30a400d4}


Constante para tratar el valor 9999 como infinito. 

\index{grafo.h@{grafo.h}!kMaxVert@{kMaxVert}}
\index{kMaxVert@{kMaxVert}!grafo.h@{grafo.h}}
\subsubsection{\setlength{\rightskip}{0pt plus 5cm}{\bf kMaxVert} = 300}\label{grafo_8h_71455579228718609881a5156b723110}


Constante con el numero maximo de vertices que tendra el grafo. 

\index{grafo.h@{grafo.h}!kNoValor@{kNoValor}}
\index{kNoValor@{kNoValor}!grafo.h@{grafo.h}}
\subsubsection{\setlength{\rightskip}{0pt plus 5cm}{\bf kNoValor} = -1}\label{grafo_8h_91cb155352045d96938a4a803e140268}


Constante para tratar el valor -1 como ausencia de valor. 

