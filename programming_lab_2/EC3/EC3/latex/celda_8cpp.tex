\section{Escritorio/EC3/src/celda.cpp File Reference}
\label{celda_8cpp}\index{Escritorio/EC3/src/celda.cpp@{Escritorio/EC3/src/celda.cpp}}
Implementacion de los metodos de la clase celda. 

{\tt \#include \char`\"{}celda.h\char`\"{}}\par
{\tt \#include \char`\"{}prision.h\char`\"{}}\par
\subsection*{Functions}
\begin{CompactItemize}
\item 
ostream \& {\bf operator$<$$<$} (ostream \&flujo, const {\bf Celda} \&C)
\end{CompactItemize}


\subsection{Detailed Description}
Implementacion de los metodos de la clase celda. 

\begin{Desc}
\item[Date:]20-04-09 \end{Desc}
\begin{Desc}
\item[Author:]{\bf Nombre:} Mario \par
 {\bf Apellidos:} Corchero Jimenez \par
 {\bf Asignatura} Laboratorio de Programacion II \par
 {\bf Curso} 08/09 \end{Desc}


\subsection{Function Documentation}
\index{celda.cpp@{celda.cpp}!operator<<@{operator$<$$<$}}
\index{operator<<@{operator$<$$<$}!celda.cpp@{celda.cpp}}
\subsubsection{\setlength{\rightskip}{0pt plus 5cm}ostream\& operator$<$$<$ (ostream \& {\em flujo}, const {\bf Celda} \& {\em C})}\label{celda_8cpp_1ae028893035e539215673ec05256b71}


Sobrecarga del operador $<$$<$ \begin{Desc}
\item[Parameters:]
\begin{description}
\item[{\em flujo}]flujo al que se introduce la informacion \item[{\em C}]celda de la que obtener la informacion \end{description}
\end{Desc}
\begin{Desc}
\item[Returns:]flujo con la informacion \end{Desc}
