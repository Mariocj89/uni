\section{Escritorio/EC3/src/grafo.cpp File Reference}
\label{grafo_8cpp}\index{Escritorio/EC3/src/grafo.cpp@{Escritorio/EC3/src/grafo.cpp}}
Implementacion de los metodos de la clase grafo. 

{\tt \#include \char`\"{}grafo.h\char`\"{}}\par
\subsection*{Functions}
\begin{CompactItemize}
\item 
int {\bf Maximo} (const int \&a, const int \&b)
\item 
int {\bf Minimo} (const int \&a, const int \&b)
\item 
ostream \& {\bf operator$<$$<$} (ostream \&flujo, const {\bf Grafo} \&G)
\end{CompactItemize}


\subsection{Detailed Description}
Implementacion de los metodos de la clase grafo. 

\begin{Desc}
\item[Author:]{\bf Nombre:} Mario \par
 {\bf Apellidos:} Corchero Jimenez \par
 {\bf Asignatura} Laboratorio de Programacion II \par
 {\bf Curso} 08/09 \end{Desc}


\subsection{Function Documentation}
\index{grafo.cpp@{grafo.cpp}!Maximo@{Maximo}}
\index{Maximo@{Maximo}!grafo.cpp@{grafo.cpp}}
\subsubsection{\setlength{\rightskip}{0pt plus 5cm}int Maximo (const int \& {\em a}, const int \& {\em b})}\label{grafo_8cpp_bd67fd90183cf2cea5f41d88b098ba97}


Funcion auxiliar que halla el máximo de dos enteros \begin{Desc}
\item[Parameters:]
\begin{description}
\item[{\em a}]primer entero \item[{\em b}]segundo entero \end{description}
\end{Desc}
\begin{Desc}
\item[Returns:]maximo de los 2, el segundo si son iguales \end{Desc}
\index{grafo.cpp@{grafo.cpp}!Minimo@{Minimo}}
\index{Minimo@{Minimo}!grafo.cpp@{grafo.cpp}}
\subsubsection{\setlength{\rightskip}{0pt plus 5cm}int Minimo (const int \& {\em a}, const int \& {\em b})}\label{grafo_8cpp_ad08456ae6cb5f9aaaee75537f6ef90a}


Funcion auxiliar que halla el mínimo de dos enteros \begin{Desc}
\item[Parameters:]
\begin{description}
\item[{\em a}]primer entero \item[{\em b}]segundo entero \end{description}
\end{Desc}
\begin{Desc}
\item[Returns:]minimo de los 2, el segundo si son iguales \end{Desc}
\index{grafo.cpp@{grafo.cpp}!operator<<@{operator$<$$<$}}
\index{operator<<@{operator$<$$<$}!grafo.cpp@{grafo.cpp}}
\subsubsection{\setlength{\rightskip}{0pt plus 5cm}ostream\& operator$<$$<$ (ostream \& {\em flujo}, const {\bf Grafo} \& {\em G})}\label{grafo_8cpp_a10916bc791b8c23a52309107d104932}


Sobrecarga del operador $<$$<$, muestra la matrid de adyacencia \begin{Desc}
\item[Parameters:]
\begin{description}
\item[{\em flujo}]flujo en el que se almacena toda la informacion \item[{\em G}]grafo a mostrar \end{description}
\end{Desc}
\begin{Desc}
\item[Returns:]flujo con toda la informacion \end{Desc}
